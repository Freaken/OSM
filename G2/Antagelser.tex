\section{Antagelser}
Vi har i vores implementation antaget at følgende gælder for de brugerprogrammer
som kan loades. Der burde laves check for dem og returneres fejl, men i stedet
opfører kernen sig på uønsket/udefineret vis, hvis det alligevel skulle ske:

\begin{itemize}
\item Pointere samt bufferstørrelser givet til systemkald er gyldige og ligger indenfor
brugerprocessernes hukommelsesområde. Dette bliver ikke tjekket og kan i værste
tilfælde udnyttes til en kernel-exploit.
\item syscall\_join skal kun kaldes af en parent. Da der ellers kan opstå
race-conditions mellem forskellige processer. Dette er vil ikke give en
kerne-fejl, men kan føre til problemer i user-space.
\item syscall\_spawn skal referere til en gyldig fil, som kan eksekveres. Hvis ikke
dette er tilfældet, laves der en {\tt KERNEL\_PANIC}.
\end{itemize}


