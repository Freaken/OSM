\section{Hjælpefunktioner}
Generelt for alle vores hjælpefunktioner er, at de alle sammen sørger for at
have spinlock'en før de ændrer på process-tabellen, samt at have den før de
læser noget der har indflydelse for de ændringer de senere vil lave.

Den største undtagelse for dette er {\tt process\_get\_current\_process\_entry}
og med denne antages det at man selvhusker at lade være med at ændre
entryen uden at have spinlocken.

\begin{itemize}
\item {\tt process\_init} initaliserer process-tabellen, devicesn for {\tt
stdin}, {\tt stdout} og {\tt stderr}, samt initialiserer idle-processen.

\item {\tt process\_spawn} starter en ny process i en anden tråd og kalder
herefter {\tt process\_start} for at loade en fil til at blive kørt i denne
tråd. Den sørger også for at sætte de relevante felter i den tilhørende {\tt
process\_table\_t} og {\tt thread\_table\_tid}.

\item {\tt process\_finish} sætter en process til ``zombie''-tilstand, dræber
dens tråd, frigiver dens resourcer og vækker eventuelle andre processer der
venter på at processen bliver færdig.

\item {\tt process\_get\_current\_process} og {\tt
process\_get\_current\_process\_entry} er meget simple funktioner som mere eller
mindre gør hvad de siger: får fat i henholdsvis det aktuelle process-id eller
rækken i {\tt process\_table} som hører til den aktuelle process.

\item {\tt process\_join} venter på at en process bliver færdig og afslutter den
derefter helt, så den ikke længere er en ``zombie''. Herefter returnerer
funktionen returværdien for processen.

\item {\tt thread\_set\_process\_id} sætter process-id'et for en tråd.
\end{itemize}
