\section{Binære søgetræer}

\subsection{\tt insert}

Den generelle idé er: søg træet igennem, find det første sted du møder
{\tt NULL}, indsæt den nye værdi. Vi har internt i gruppen dog implementeret
denne idé på flere måde - de mest oplagte er at søge træet igennem med et
{\tt while}-loop eller ved hjælp af rekursion. Vi har i den afleverede opgave
brugt rekursion.

\subsection{\tt print\_inorder}

Den generelle idé her at man rekusivt først printer venstre del-træ,
så printer elementet, derefter printer det højre deltræ. For at få udskrevet et
linjeskift har vi pakket dette ind i en hjælpe-funktion.

\subsection{\tt size}

Opgaven er triviel, hvis man observerer at størrelsen af træet er 0 for et
tomt træ, mens den er 1 + størrelsen af hver af børnene for ikke-tomme
træer.

\subsection{\tt to\_array}

Opgaven løses ved at først allokere plads til array ved at bruge {\tt malloc} og
{\tt size}. Derefter udfyldes arrayet ved at gå rekursivt igennem træet. Den
aktuelle position holder vi styr på ved hjælp af en pege-peger.
